%% Generated by Sphinx.
\def\sphinxdocclass{report}
\documentclass[letterpaper,10pt,english]{sphinxmanual}
\ifdefined\pdfpxdimen
   \let\sphinxpxdimen\pdfpxdimen\else\newdimen\sphinxpxdimen
\fi \sphinxpxdimen=.75bp\relax

\PassOptionsToPackage{warn}{textcomp}
\usepackage[utf8]{inputenc}
\ifdefined\DeclareUnicodeCharacter
% support both utf8 and utf8x syntaxes
  \ifdefined\DeclareUnicodeCharacterAsOptional
    \def\sphinxDUC#1{\DeclareUnicodeCharacter{"#1}}
  \else
    \let\sphinxDUC\DeclareUnicodeCharacter
  \fi
  \sphinxDUC{00A0}{\nobreakspace}
  \sphinxDUC{2500}{\sphinxunichar{2500}}
  \sphinxDUC{2502}{\sphinxunichar{2502}}
  \sphinxDUC{2514}{\sphinxunichar{2514}}
  \sphinxDUC{251C}{\sphinxunichar{251C}}
  \sphinxDUC{2572}{\textbackslash}
\fi
\usepackage{cmap}
\usepackage[T1]{fontenc}
\usepackage{amsmath,amssymb,amstext}
\usepackage{babel}



\usepackage{times}
\expandafter\ifx\csname T@LGR\endcsname\relax
\else
% LGR was declared as font encoding
  \substitutefont{LGR}{\rmdefault}{cmr}
  \substitutefont{LGR}{\sfdefault}{cmss}
  \substitutefont{LGR}{\ttdefault}{cmtt}
\fi
\expandafter\ifx\csname T@X2\endcsname\relax
  \expandafter\ifx\csname T@T2A\endcsname\relax
  \else
  % T2A was declared as font encoding
    \substitutefont{T2A}{\rmdefault}{cmr}
    \substitutefont{T2A}{\sfdefault}{cmss}
    \substitutefont{T2A}{\ttdefault}{cmtt}
  \fi
\else
% X2 was declared as font encoding
  \substitutefont{X2}{\rmdefault}{cmr}
  \substitutefont{X2}{\sfdefault}{cmss}
  \substitutefont{X2}{\ttdefault}{cmtt}
\fi


\usepackage[Bjarne]{fncychap}
\usepackage{sphinx}

\fvset{fontsize=\small}
\usepackage{geometry}


% Include hyperref last.
\usepackage{hyperref}
% Fix anchor placement for figures with captions.
\usepackage{hypcap}% it must be loaded after hyperref.
% Set up styles of URL: it should be placed after hyperref.
\urlstyle{same}


\usepackage{sphinxmessages}




\title{ACSE\_la}
\date{Oct 30, 2020}
\release{}
\author{Jakob Torben}
\newcommand{\sphinxlogo}{\vbox{}}
\renewcommand{\releasename}{}
\makeindex
\begin{document}

\ifdefined\shorthandoff
  \ifnum\catcode`\=\string=\active\shorthandoff{=}\fi
  \ifnum\catcode`\"=\active\shorthandoff{"}\fi
\fi

\pagestyle{empty}
\sphinxmaketitle
\pagestyle{plain}
\sphinxtableofcontents
\pagestyle{normal}
\phantomsection\label{\detokenize{index::doc}}


\sphinxcode{\sphinxupquote{acse\_la}} is a linear algebra library that implements Gaussian Elimination, matrix multiplication and calculating the determinant.


\chapter{A Gaussian Elimination routine}
\label{\detokenize{index:a-gaussian-elimination-routine}}
This package implements Gaussian elimination {[}1{]} for list of lists and \sphinxcode{\sphinxupquote{numpy.ndarray}} objects, along with hand\sphinxhyphen{}written matrix multiplication.

See \sphinxcode{\sphinxupquote{acse\_la.gauss()}} and \sphinxcode{\sphinxupquote{acse\_la.gauss.matmul()}} for more information.


\begin{fulllineitems}
\pysiglinewithargsret{\sphinxcode{\sphinxupquote{acse\_la.gauss.}}\sphinxbfcode{\sphinxupquote{gauss}}}{\emph{\DUrole{n}{a}}, \emph{\DUrole{n}{b}}}{}
Given two matrices, \sphinxtitleref{a} and \sphinxtitleref{b}, with \sphinxtitleref{a} square, the determinant
of \sphinxtitleref{a} and a matrix \sphinxtitleref{x} such that a*x = b are returned.
If \sphinxtitleref{b} is the identity, then \sphinxtitleref{x} is the inverse of \sphinxtitleref{a}.
\begin{quote}\begin{description}
\item[{Parameters}] \leavevmode\begin{itemize}
\item {} 
\sphinxstyleliteralstrong{\sphinxupquote{a}} (\sphinxstyleliteralemphasis{\sphinxupquote{np.array}}\sphinxstyleliteralemphasis{\sphinxupquote{ or }}\sphinxstyleliteralemphasis{\sphinxupquote{list of lists}}) \textendash{} ‘n x n’ array

\item {} 
\sphinxstyleliteralstrong{\sphinxupquote{b}} (\sphinxstyleliteralemphasis{\sphinxupquote{np. array}}\sphinxstyleliteralemphasis{\sphinxupquote{ or }}\sphinxstyleliteralemphasis{\sphinxupquote{list of lists}}) \textendash{} ‘m x n’ array

\end{itemize}

\end{description}\end{quote}
\subsubsection*{Examples}

\begin{sphinxVerbatim}[commandchars=\\\{\}]
\PYG{g+gp}{\PYGZgt{}\PYGZgt{}\PYGZgt{} }\PYG{n}{a} \PYG{o}{=} \PYG{p}{[}\PYG{p}{[}\PYG{l+m+mi}{2}\PYG{p}{,} \PYG{l+m+mi}{0}\PYG{p}{,} \PYG{o}{\PYGZhy{}}\PYG{l+m+mi}{1}\PYG{p}{]}\PYG{p}{,} \PYG{p}{[}\PYG{l+m+mi}{0}\PYG{p}{,} \PYG{l+m+mi}{5}\PYG{p}{,} \PYG{l+m+mi}{6}\PYG{p}{]}\PYG{p}{,} \PYG{p}{[}\PYG{l+m+mi}{0}\PYG{p}{,} \PYG{o}{\PYGZhy{}}\PYG{l+m+mi}{1}\PYG{p}{,} \PYG{l+m+mi}{1}\PYG{p}{]}\PYG{p}{]}
\PYG{g+gp}{\PYGZgt{}\PYGZgt{}\PYGZgt{} }\PYG{n}{b} \PYG{o}{=} \PYG{p}{[}\PYG{p}{[}\PYG{l+m+mi}{2}\PYG{p}{]}\PYG{p}{,} \PYG{p}{[}\PYG{l+m+mi}{1}\PYG{p}{]}\PYG{p}{,} \PYG{p}{[}\PYG{l+m+mi}{2}\PYG{p}{]}\PYG{p}{]}
\PYG{g+gp}{\PYGZgt{}\PYGZgt{}\PYGZgt{} }\PYG{n}{det}\PYG{p}{,} \PYG{n}{x} \PYG{o}{=} \PYG{n}{gauss}\PYG{p}{(}\PYG{n}{a}\PYG{p}{,} \PYG{n}{b}\PYG{p}{)}
\PYG{g+gp}{\PYGZgt{}\PYGZgt{}\PYGZgt{} }\PYG{n}{det}
\PYG{g+go}{22.0}
\PYG{g+gp}{\PYGZgt{}\PYGZgt{}\PYGZgt{} }\PYG{n}{x}
\PYG{g+go}{[[1.5], [\PYGZhy{}1.0], [1.0]]}
\PYG{g+gp}{\PYGZgt{}\PYGZgt{}\PYGZgt{} }\PYG{n}{A} \PYG{o}{=} \PYG{p}{[}\PYG{p}{[}\PYG{l+m+mi}{1}\PYG{p}{,} \PYG{l+m+mi}{0}\PYG{p}{,} \PYG{o}{\PYGZhy{}}\PYG{l+m+mi}{1}\PYG{p}{]}\PYG{p}{,} \PYG{p}{[}\PYG{o}{\PYGZhy{}}\PYG{l+m+mi}{2}\PYG{p}{,} \PYG{l+m+mi}{3}\PYG{p}{,} \PYG{l+m+mi}{0}\PYG{p}{]}\PYG{p}{,} \PYG{p}{[}\PYG{l+m+mi}{1}\PYG{p}{,} \PYG{o}{\PYGZhy{}}\PYG{l+m+mi}{3}\PYG{p}{,} \PYG{l+m+mi}{2}\PYG{p}{]}\PYG{p}{]}
\PYG{g+gp}{\PYGZgt{}\PYGZgt{}\PYGZgt{} }\PYG{n}{I} \PYG{o}{=} \PYG{p}{[}\PYG{p}{[}\PYG{l+m+mi}{1}\PYG{p}{,} \PYG{l+m+mi}{0}\PYG{p}{,} \PYG{l+m+mi}{0}\PYG{p}{]}\PYG{p}{,} \PYG{p}{[}\PYG{l+m+mi}{0}\PYG{p}{,} \PYG{l+m+mi}{1}\PYG{p}{,} \PYG{l+m+mi}{0}\PYG{p}{]}\PYG{p}{,} \PYG{p}{[}\PYG{l+m+mi}{0}\PYG{p}{,} \PYG{l+m+mi}{0}\PYG{p}{,} \PYG{l+m+mi}{1}\PYG{p}{]}\PYG{p}{]}
\PYG{g+gp}{\PYGZgt{}\PYGZgt{}\PYGZgt{} }\PYG{n}{Det}\PYG{p}{,} \PYG{n}{Ainv} \PYG{o}{=} \PYG{n}{gauss}\PYG{p}{(}\PYG{n}{A}\PYG{p}{,} \PYG{n}{I}\PYG{p}{)}
\PYG{g+gp}{\PYGZgt{}\PYGZgt{}\PYGZgt{} }\PYG{n}{Det}
\PYG{g+go}{3.0}
\PYG{g+gp}{\PYGZgt{}\PYGZgt{}\PYGZgt{} }\PYG{n}{Ainv} 
\PYG{g+go}{[[2.0, 1.0, 1.0],}
\PYG{g+go}{[1.3333333333333333, 1.0, 0.6666666666666666],}
\PYG{g+go}{[1.0, 1.0, 1.0]]}
\end{sphinxVerbatim}
\subsubsection*{Notes}

See \sphinxurl{https://en.wikipedia.org/wiki/Gaussian\_elimination} for further details.

\end{fulllineitems}



\begin{fulllineitems}
\pysiglinewithargsret{\sphinxcode{\sphinxupquote{acse\_la.gauss.}}\sphinxbfcode{\sphinxupquote{matmul}}}{\emph{\DUrole{n}{a}}, \emph{\DUrole{n}{b}}}{}
Matrix product of matrix a and  matrix b.
\begin{quote}\begin{description}
\item[{Parameters}] \leavevmode\begin{itemize}
\item {} 
\sphinxstyleliteralstrong{\sphinxupquote{a}} (\sphinxstyleliteralemphasis{\sphinxupquote{np.array}}\sphinxstyleliteralemphasis{\sphinxupquote{ or }}\sphinxstyleliteralemphasis{\sphinxupquote{list of lists}}) \textendash{} ‘n x m’ array

\item {} 
\sphinxstyleliteralstrong{\sphinxupquote{b}} (\sphinxstyleliteralemphasis{\sphinxupquote{np. array}}\sphinxstyleliteralemphasis{\sphinxupquote{ or }}\sphinxstyleliteralemphasis{\sphinxupquote{list of lists}}) \textendash{} ‘m x l’ array

\end{itemize}

\item[{Returns}] \leavevmode
\sphinxstylestrong{out} \textendash{} The matrix product of the inputs.

\item[{Return type}] \leavevmode
list of lists

\item[{Raises}] \leavevmode
\sphinxstyleliteralstrong{\sphinxupquote{ValueError}} \textendash{} If the number of columns of \sphinxtitleref{a} is not the same as
    the number of rows \sphinxtitleref{b}.

\end{description}\end{quote}
\subsubsection*{Notes}

The output dimension depends on the dimensions of the input.
\begin{itemize}
\item {} 
For input matrices a = n x m and b = m x l, the output matrix will have
dimensions n x l.

\end{itemize}
\subsubsection*{Examples}

For 2\sphinxhyphen{}D arrays:

\begin{sphinxVerbatim}[commandchars=\\\{\}]
\PYG{g+gp}{\PYGZgt{}\PYGZgt{}\PYGZgt{} }\PYG{n}{a} \PYG{o}{=} \PYG{n}{np}\PYG{o}{.}\PYG{n}{array}\PYG{p}{(}\PYG{p}{[}\PYG{p}{[}\PYG{l+m+mi}{1}\PYG{p}{,} \PYG{l+m+mi}{2}\PYG{p}{]}\PYG{p}{,}
\PYG{g+gp}{... }              \PYG{p}{[}\PYG{l+m+mi}{3}\PYG{p}{,} \PYG{l+m+mi}{4}\PYG{p}{]}\PYG{p}{]}\PYG{p}{)}
\PYG{g+gp}{\PYGZgt{}\PYGZgt{}\PYGZgt{} }\PYG{n}{b} \PYG{o}{=} \PYG{n}{np}\PYG{o}{.}\PYG{n}{array}\PYG{p}{(}\PYG{p}{[}\PYG{p}{[}\PYG{l+m+mi}{5}\PYG{p}{,} \PYG{l+m+mi}{1}\PYG{p}{]}\PYG{p}{,}
\PYG{g+gp}{... }              \PYG{p}{[}\PYG{l+m+mi}{6}\PYG{p}{,} \PYG{l+m+mi}{2}\PYG{p}{]}\PYG{p}{]}\PYG{p}{)}
\PYG{g+gp}{\PYGZgt{}\PYGZgt{}\PYGZgt{} }\PYG{n}{matmul}\PYG{p}{(}\PYG{n}{a}\PYG{p}{,} \PYG{n}{b}\PYG{p}{)}
\PYG{g+go}{[[17, 5], [39, 11]]}
\end{sphinxVerbatim}

For a 2\sphinxhyphen{}D array and 1\sphinxhyphen{}D array:
\textgreater{}\textgreater{}\textgreater{} a = np.array({[}{[}1, 2{]},
…               {[}3, 4{]}{]})
\textgreater{}\textgreater{}\textgreater{} b = np.array({[}5, 6{]})
\textgreater{}\textgreater{}\textgreater{} matmul(a, b)
{[}17, 39{]}

\end{fulllineitems}



\begin{fulllineitems}
\pysiglinewithargsret{\sphinxcode{\sphinxupquote{acse\_la.gauss.}}\sphinxbfcode{\sphinxupquote{zeromat}}}{\emph{\DUrole{n}{a}}, \emph{\DUrole{n}{b}}}{}
Returns an array with dimension shape, filled with zeros.
\begin{quote}\begin{description}
\item[{Parameters}] \leavevmode
\sphinxstyleliteralstrong{\sphinxupquote{shape}} (\sphinxstyleliteralemphasis{\sphinxupquote{tuple}}) \textendash{} Shape of output matrix

\item[{Returns}] \leavevmode
\sphinxstylestrong{out} \textendash{} Matrix with dimension p x q

\item[{Return type}] \leavevmode
list of lists

\end{description}\end{quote}
\subsubsection*{Examples}

\begin{sphinxVerbatim}[commandchars=\\\{\}]
\PYG{g+gp}{\PYGZgt{}\PYGZgt{}\PYGZgt{} }\PYG{n}{zeromat}\PYG{p}{(}\PYG{l+m+mi}{2}\PYG{p}{,} \PYG{l+m+mi}{3}\PYG{p}{)}
\PYG{g+go}{[[0, 0, 0], [0, 0, 0]]}
\PYG{g+gp}{\PYGZgt{}\PYGZgt{}\PYGZgt{} }\PYG{n}{zeromat}\PYG{p}{(}\PYG{l+m+mi}{1}\PYG{p}{,} \PYG{l+m+mi}{3}\PYG{p}{)}
\PYG{g+go}{[[0, 0, 0]]}
\PYG{g+gp}{\PYGZgt{}\PYGZgt{}\PYGZgt{} }\PYG{n}{zeromat}\PYG{p}{(}\PYG{l+m+mi}{2}\PYG{p}{,} \PYG{l+m+mi}{1}\PYG{p}{)}
\PYG{g+go}{[[0], [0]]}
\end{sphinxVerbatim}

\end{fulllineitems}



\chapter{A determinant routine}
\label{\detokenize{index:a-determinant-routine}}

\begin{fulllineitems}
\pysiglinewithargsret{\sphinxcode{\sphinxupquote{acse\_la.det.}}\sphinxbfcode{\sphinxupquote{det}}}{\emph{\DUrole{n}{A}}}{}
Compute the determinant of a matrix.
\begin{quote}\begin{description}
\item[{Parameters}] \leavevmode
\sphinxstyleliteralstrong{\sphinxupquote{A}} (\sphinxstyleliteralemphasis{\sphinxupquote{np.array}}\sphinxstyleliteralemphasis{\sphinxupquote{ or }}\sphinxstyleliteralemphasis{\sphinxupquote{list of lists}}) \textendash{} ‘N x N’ matrix

\item[{Returns}] \leavevmode
\sphinxstylestrong{out} \textendash{} The determinant of the matrix.

\item[{Return type}] \leavevmode
float

\item[{Raises}] \leavevmode
\sphinxstyleliteralstrong{\sphinxupquote{ValueError}} \textendash{} If input is not square matrix

\end{description}\end{quote}
\subsubsection*{Notes}
\begin{itemize}
\item {} 
The determinant is computed with LU decomposition, using Crout’s method.
For further details see: Propp, J. G., Wilson, D. B. ‘Numerical Recipes’,
Cambridge University Press. (1996).

\item {} 
Pivoting is not yet implemented. Matrices with zero entries along
diagonal can have unstable behaviour.

\end{itemize}
\subsubsection*{Examples}

\begin{sphinxVerbatim}[commandchars=\\\{\}]
\PYG{g+gp}{\PYGZgt{}\PYGZgt{}\PYGZgt{} }\PYG{n}{det}\PYG{p}{(}\PYG{p}{[}\PYG{p}{[}\PYG{l+m+mi}{2}\PYG{p}{,} \PYG{l+m+mi}{9}\PYG{p}{,} \PYG{l+m+mi}{4}\PYG{p}{]}\PYG{p}{,} \PYG{p}{[}\PYG{l+m+mi}{7}\PYG{p}{,} \PYG{l+m+mi}{5}\PYG{p}{,} \PYG{l+m+mi}{3}\PYG{p}{]}\PYG{p}{,} \PYG{p}{[}\PYG{l+m+mi}{6}\PYG{p}{,} \PYG{l+m+mi}{1}\PYG{p}{,} \PYG{l+m+mi}{8}\PYG{p}{]}\PYG{p}{]}\PYG{p}{)}
\PYG{g+go}{\PYGZhy{}360.0}
\PYG{g+gp}{\PYGZgt{}\PYGZgt{}\PYGZgt{} }\PYG{n}{det}\PYG{p}{(}\PYG{p}{[}\PYG{p}{[}\PYG{l+m+mf}{0.5}\PYG{p}{,} \PYG{l+m+mf}{1.5}\PYG{p}{]}\PYG{p}{,} \PYG{p}{[}\PYG{l+m+mf}{4.2}\PYG{p}{,} \PYG{l+m+mf}{3.9}\PYG{p}{]}\PYG{p}{]}\PYG{p}{)}
\PYG{g+go}{\PYGZhy{}4.35}
\end{sphinxVerbatim}

\end{fulllineitems}



\chapter{Citation}
\label{\detokenize{index:citation}}
{[}1{]} \sphinxurl{https://mathworld.wolfram.com/GaussianElimination.html}

{[}2{]} Propp, J. G., Wilson, D. B. ‘Numerical Recipes’, Cambridge University Press. (1996)



\renewcommand{\indexname}{Index}
\printindex
\end{document}